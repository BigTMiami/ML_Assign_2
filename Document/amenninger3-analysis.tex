\def\year{2021}\relax
%File: formatting-instructions-latex-2021.tex
%release 2021.1
\documentclass[letterpaper]{article} % DO NOT CHANGE THIS
\usepackage{aaai21}  % DO NOT CHANGE THIS
\usepackage{times}  % DO NOT CHANGE THIS
\usepackage{helvet} % DO NOT CHANGE THIS
\usepackage{courier}  % DO NOT CHANGE THIS
\usepackage[hyphens]{url}  % DO NOT CHANGE THIS
\usepackage{graphicx} % DO NOT CHANGE THIS

% ADJUSTED BY AFM ----------------------------------------------------------
\usepackage{amsmath} % ENTERED BY AFM

\urlstyle{rm} % DO NOT CHANGE THIS
\def\UrlFont{\rm}  % DO NOT CHANGE THIS
\usepackage{natbib}  % DO NOT CHANGE THIS AND DO NOT ADD ANY OPTIONS TO IT
\usepackage{caption} % DO NOT CHANGE THIS AND DO NOT ADD ANY OPTIONS TO IT
\frenchspacing  % DO NOT CHANGE THIS
\setlength{\pdfpagewidth}{8.5in}  % DO NOT CHANGE THIS
\setlength{\pdfpageheight}{11in}  % DO NOT CHANGE THIS
\nocopyright
%PDF Info Is REQUIRED.
% For /Author, add all authors within the parentheses, separated by commas. No accents or commands.
% For /Title, add Title in Mixed Case. No accents or commands. Retain the parentheses.
\pdfinfo{
/Title (ML Assignment 1)
/Author (Anthony Menninger)
/TemplateVersion (2021.1)
} %Leave this
% /Title ()
% Put your actual complete title (no codes, scripts, shortcuts, or LaTeX commands) within the parentheses in mixed case
% Leave the space between \Title and the beginning parenthesis alone
% /Author ()
% Put your actual complete list of authors (no codes, scripts, shortcuts, or LaTeX commands) within the parentheses in mixed case.
% Each author should be only by a comma. If the name contains accents, remove them. If there are any LaTeX commands,
% remove them.

% DISALLOWED PACKAGES
% \usepackage{authblk} -- This package is specifically forbidden
% \usepackage{balance} -- This package is specifically forbidden
% \usepackage{color (if used in text)
% \usepackage{CJK} -- This package is specifically forbidden
% \usepackage{float} -- This package is specifically forbidden
% \usepackage{flushend} -- This package is specifically forbidden
% \usepackage{fontenc} -- This package is specifically forbidden
% \usepackage{fullpage} -- This package is specifically forbidden
% \usepackage{geometry} -- This package is specifically forbidden
% \usepackage{grffile} -- This package is specifically forbidden
% \usepackage{hyperref} -- This package is specifically forbidden
% \usepackage{navigator} -- This package is specifically forbidden
% (or any other package that embeds links such as navigator or hyperref)
% \indentfirst} -- This package is specifically forbidden
% \layout} -- This package is specifically forbidden
% \multicol} -- This package is specifically forbidden
% \nameref} -- This package is specifically forbidden
% \usepackage{savetrees} -- This package is specifically forbidden
% \usepackage{setspace} -- This package is specifically forbidden
% \usepackage{stfloats} -- This package is specifically forbidden
% \usepackage{tabu} -- This package is specifically forbidden
% \usepackage{titlesec} -- This package is specifically forbidden
% \usepackage{tocbibind} -- This package is specifically forbidden
% \usepackage{ulem} -- This package is specifically forbidden
% \usepackage{wrapfig} -- This package is specifically forbidden
% DISALLOWED COMMANDS
% \nocopyright -- Your paper will not be published if you use this command
% \addtolength -- This command may not be used
% \balance -- This command may not be used
% \baselinestretch -- Your paper will not be published if you use this command
% \clearpage -- No page breaks of any kind may be used for the final version of your paper
% \columnsep -- This command may not be used
% \newpage -- No page breaks of any kind may be used for the final version of your paper
% \pagebreak -- No page breaks of any kind may be used for the final version of your paperr
% \pagestyle -- This command may not be used
% \tiny -- This is not an acceptable font size.
% \vspace{- -- No negative value may be used in proximity of a caption, figure, table, section, subsection, subsubsection, or reference
% \vskip{- -- No negative value may be used to alter spacing above or below a caption, figure, table, section, subsection, subsubsection, or reference

\setcounter{secnumdepth}{0} %May be changed to 1 or 2 if section numbers are desired.

% The file aaai21.sty is the style file for AAAI Press
% proceedings, working notes, and technical reports.
%

% Title

% Your title must be in mixed case, not sentence case.
% That means all verbs (including short verbs like be, is, using,and go),
% nouns, adverbs, adjectives should be capitalized, including both words in hyphenated terms, while
% articles, conjunctions, and prepositions are lower case unless they
% directly follow a colon or long dash


%Example, Single Author, ->> remove \iffalse,\fi and place them surrounding AAAI title to use it
\title{
Machine Learning - CS 7641
Assignment 2
	
}
\author {
    % Author
    Anthony Menninger \\
}

\affiliations{
    Georgia Tech OMSCS Program \\
    amenninger3\\
    tmenninger@gatech.edu

}

\begin{document}

\maketitle

\begin{abstract}
This paper first reviews three optimization problems using four different algorithms. The algorithms considered are Random Hill Climbing, Simulated Annealing, Genetic Algorithm and MIMIC.  The Four Peaks problem highlights Genetic Algorithm, the K Colors problem highlights MIMIC and the Knapsack Problem highlights Simulated Annealing.  The paper then reviews using Random Hill Climbing, Simulated Annealing, Genetic Algorithm for learning neural network weights for the MNIST image dataset and compares these results to using Back propagation.  
\end{abstract}

\section{Introduction}
The core library used in the experiments in this paper is mlrose [3].  This was developed by Georgia Tech students to support this class by providing a comprehensive tools set for reviewing randomized optimization.  I used the mlrose-hiive [2] fork, which has had other students fix bugs and enhance functionality.





\section{Four Peaks}

\section{Knapsack}

\section{K Colors}


\section{Neural Network Optimization}

The data set is The MNIST Database of Handwritten Digits [2].  This consists of instances of 28 X 28 greyscale images of the digits 0-9.  One transformation performed was that the values where scaled to 0 to 1, from 0 to 255.  For all but the neural network algorithm,  the images were flattened, creating 784 features, one for each pixel.  There are 60,000 training instances and 10,000 test instances.  I chose this because it is a good comparison to the first data set, with a very different type of data (images), which leads to significantly more features (784).  In addition, each of the features can be thought of as related to each other, as they are all positional in a two dimensional grid, while the first data set features do not have any necessary relation between themselves ie:  age is not related to sex.  

Did use Pytorch, run.  Switched over to mlrose


\section{References}
\begin{tabular}{l p{2.75in}}
\\
1 & The MNIST Database of Handwritten Digits. url: http://yann.lecun.com/exdb/mnist/.
\\
2 & Rollings, A. (2020). mlrose: Machine Learning, Randomized Optimization and SEarch package for Python, hiive extended remix. https://github.com/hiive/mlrose. Accessed: 2/11/2022
\\
3 & Hayes, G. (2019). mlrose: Machine Learning, Randomized Optimization and SEarch package for Python. https://github.com/gkhayes/mlrose. Accessed: 2/11/2022.
\end{tabular}
\end{document}
